\chapter{绪论}
% 总起一段话

情感分析是一个热门的研究课题,情感分析的研究场景从文本到多模态,研究对象从句子级到属性级。在技术迭代发展迅速的今天,情感分析的研究也经历了从传统统计模型到深度学习模型的过渡。本课题从图文关系的视角出发,研究多模态情感分析问题,旨在准确识别多模态样本(图文对)蕴含的情感。针对与现有图文关系中忽略了隐含的图文情感关联的问题,因此本课题聚焦图文关系角度去帮助情感分析。本章主要从以下几个方面具体介绍基于图文关系的多模态方面级情感分析研究:首先,介绍本课题的研究背景与意义;随后,对国内外相关领域的研究工作进行概述,阐明本研究与前人工作的关联与区别;然后,明晰本课题的研究逻辑,并给出具体的研究内容;最后,明确本文的组织结构。


\section{研究背景}
情感分析(Sentiment Analysis)作为自然语言处理(Natural Language Processing, NLP)领域的重要研究方向,旨在识别与理解文本中所蕴含的主观情绪与态度。早期研究主要集中于基于文本的情感分析,通过情感词典、统计学习或深度学习模型对文本进行情感极性分类。然而,随着社交媒体和移动互联网的迅猛发展,人们在表达情感和观点时逐渐采用更加丰富的多模态形式,如文字、图片、视频等,仅依赖单一文本模态的情感分析方法已难以全面刻画用户的真实情绪。多模态情感分析(Multimodal Sentiment Analysis, MSA)因此应运而生,通过融合不同模态的信息实现情感的综合识别与推理,成为近年来学术界和工业界的研究热点。同时在现实生活中,社交媒体的兴起极大地改变了人类的信息传播与情绪表达方式。微博、微信、Twitter、Facebook 等社交平台已成为人们分享生活、发表观点和参与公共讨论的重要渠道。相较于传统媒体,社交媒体具有传播速度快、内容形式多样、情绪表达自由度高等显著特征。用户在这些平台上生成了大量包含文本与图像的多模态数据,这些信息不仅反映了社会舆情的动态变化,也记录了公众在不同事件背景下的情感波动。如何从这些庞杂的多模态数据中挖掘出真实、细腻的情感信息,对于政府和相关机构开展舆情监测、风险预警与干预具有重要意义。情感分析技术能够帮助管理部门及时识别社会情绪的走向,为政策制定、品牌管理和公共事件应对提供数据支撑,从而在社会治理和公共安全等领域发挥关键作用。
然而,社交媒体环境下的多模态情感分析仍面临诸多挑战。与传统的文本情感分析不同,社交媒体中的图文内容往往存在情感指向模糊及噪声信息干扰等问题。图片与文本之间可能相互补充、矛盾甚至完全无关,导致模型在融合不同模态信息时容易受到无关信息的干扰。此外,现有研究通常仅关注图文间的语义匹配关系,而缺乏对图文“情感相关性”的明确定义与建模,难以区分情感相关与仅语义相关的图文配对关系。这一问题使得模型在面对复杂的社交媒体内容时,难以准确捕捉隐藏于图像与文字交互背后的情感意图,从而限制了其在真实应用场景中的鲁棒性与解释性。
针对上述问题,本研究提出了一种基于图文关系的多模态情感分析框架,旨在从情感表达角度刻画图文之间的关联。我们首先定义了基于情感表达的方面级图文相关性,并在此基础上人工标注了 3562 条数据,构建了首个方面级图文相关性数据集,以弥补现有数据在情感关系层面上的缺失。随后,我们对该数据集进行了系统的数据分析与基准验证实验,揭示了模型在不同图文关系类型下的性能差异。进一步地,我们引入大型语言模型(Large Language Model, LLM)从图片和文本中提取情感线索,将其作为补充信息用于细粒度情感控制,从而提升模型对情感表达一致性的理解能力。最后,考虑到人工标注数据规模有限的问题,我们探索了利用 LLM 扩充方面级图文情感相关性数据集的可行性,并设计了多模态预训练任务,使模型在更大规模的弱监督数据上获得图文关系判断与情感关联建模的能力。综上所述,本研究通过构建情感相关性数据集、引入情感线索增强与多模态预训练机制,旨在提升模型对社交媒体复杂图文关系的理解与推理能力,从而为多模态情感分析提供新的研究视角与方法支撑。


\section{研究意义}
% 直接从两个角度来切入,学术价值和应用价值
% 学术价值,可以从多模态情感分析对于科学研究作用,比如它的地位,然后可以描述下对其他任务的帮助
% 应用价值,可以多模态情感分析在社交媒体等场景中的应用价值,比如可以帮助政府和相关机构及时识别社会情绪的走向,为政策制定、品牌管理和公共事件应对提供数据支撑,同时也可以为社交媒体平台的内容推荐、情感分析等服务提供技术支持。例如智能客服系统可以根据用户的文本和图像描述,及时识别用户的情感状态,为客户提供个性化的服务。
多模态情感分析作为跨模态领域的基础任务,其目的是充分理解对用户情感状态的准确识别与推理。其在学术方面和应用方面都具有重要意义。

从学术价值角度,多模态方面级情感分析作为交叉领域的前沿课题,能够推动多模态认知智能的发展以及为相关下游任务提供理论范式支撑。首先传统的单模态情感分析往往受限于信息维度的单一性,难以捕捉用户在图文场景下的真实意图;而粗粒度的多模态句子级情感分析虽然融合了视觉信息,却忽略了在同一句子中针对不同评价方面可能存在的矛盾情感(例如:“这就餐厅环境很美,但服务太差了”配以精美环境图)。本研究通过聚焦方面级这一细粒度维度,深入探索图文多模态数在特定实体的情感关联。这有助于实现对复杂人类情感的精准建模,从而提升模型在复杂场景下的认知理解能力。其次多模态情感分析能够为多模态嘲讽检测、多模态虚假信息识别等其他相关任务提供有益的技术参考和理论借鉴。

从应用价值角度,多模态方面级情感分析能够将海量的非结构化数据转化为高价值的结构化信息,在社会治理、商业决策及智能服务等领域发挥着不可替代的作用。例如,在社交媒体上,公众对于突发公共事件、政府政策或社会热点的讨论往往是图文结合的。多模态方面级情感分析技术能够帮助政府及相关监管机构从海量数据中精准定位公众情绪的爆发点,在突发灾害或公共卫生事件中,系统不仅能识别整体舆论的负面情绪,还能具体识别出民众是针对“救援速度”还是“信息透明度”等具体方面产生不满。这种细粒度的洞察力能够为政府部门提供数据支撑,辅助其及时调整策略,从源头疏导社会情绪,实现精准施策,提升公共危机应对的效率与水平。另外在互联网服务中,多模态方面级情感分析能够显著提升用户体验。例如在智能客服领域,当用户上传一张故障产品图片并附带文字描述时,系统可以准确识别出用户关注的故障“部件以及焦急或愤怒的负面情感。这使得智能客服能够优先处理高情绪强度的投诉,并根据具体问题自动匹配解决方案,实现共情回答。

综上所述,作为一项跨模态情感分析任务,多模态方面级情感分析在学术价值和应用价值上均具有重要意义。

\section{国内研究现状}
本课题主要针对基于图文关系的多模态方面级情感分析任务展开研究。而如何使用图文关系的理解来帮助多模态方面级情感分析是本课题研究的核心。因此本节主要介绍现有方面级情感分析和图文关系推理的研究现状。

\subsection{方面级情感分析}
方面级情感分析是情感分析领域的细粒度任务,其目标是识别文本或多模态内容中针对特定实体、属性、特征的情感倾向(如积极、消极、中性),区别于整体情感分析,能更精准地捕捉用户对不同焦点的差异化态度。根据数据模态的差异,可分为文本方面级情感分析与图文方面级情感分析两类。

\subsubsection{文本方面级情感分析}
文本方面级情感分析是以文本为输入,针对文本中的特定方面挖掘其对应的情感极性的任务。现有工作主要围绕注意力机制、预训练模型与图神经网络。首先,基于注意力机制的方法通过注意力权重分配聚焦与目标方面相关的上下文信息,精准捕捉局部情感特征,例如Wang等人设计了面向特定方面的注意力建模机制,实现了对不同方面的差异化聚焦,有效提升了情感极性的区分能力[1],而Chen等人针对长文本中情感特征传递不明确的问题,提出多尺度注意力机制,增强模型对长距离上下文情感关联的感知能力[2],进一步优化了复杂句式下的情感识别效果。另一方面,预训练语言模型凭借强大的上下文语义表示能力推动了该领域的性能突破,Sun等人创新性地将方面情感识别任务转化为句子对分类任务,通过将“文本-方面”构造为输入对,借助BERT模型的深层语义编码能力显著提升了情感分类的准确性[3]。最后,图神经网络的结构化建模能力则为捕捉文本中的句法依赖与语义关联提供了有效工具,Tian等人提出结合依存句法类型的图卷积模型,通过构建句法依存图将语法结构信息融入情感特征学习过程[4],Zhang等人则融合语义相似性与句法依存关系构建异构图,增强模型对目标方面的语义感知能力,有效缓解了歧义上下文带来的干扰[5]。

\subsubsection{图文对方面级情感分析}
图文对方面级情感分析是指以文本-图像对为输入数据,针对文本中指定的目标方面,综合文本与图像信息,共同推断该方面对应情感极性的任务,该任务的核心是需要充分理解并利用两种模态的有效信息,同时过滤无关模态带来的噪声干扰,现有研究可以分为以下几个部分。首先,注意力机制驱动的跨模态特征对齐方法延续文本场景的核心思想,通过构建文本-图像注意力交互模块,利用文本上下文特征引导图像相关区域的特征提取。近期,Xu等人设计了文本方面感知的视觉注意力机制,基于文本上下文动态聚焦图像中与目标方面相关的视觉区域[6]。之后,Yu等人则提出双向注意力融合框架,实现文本情感特征与图像视觉特征的相互引导与增强[7,8],提升了跨模态情感信息的一致性建模效果。另外为降低跨模态融合的技术复杂度,部分研究采用图像文本化转换策略,将图像转换为文本描述后利用成熟的文本情感分析技术完成任务,为此,Wang等人通过图像描述生成模型将视觉内容转化为结构化文本实现情感判断[9],受其启发,Xiao等人与Zhao等人分别优化了图像描述的细粒度与情感倾向性,使转换后的文本更精准地保留图像中的情感相关信息[10,11],进一步缩小了视觉与文本模态的语义鸿沟。除了将图像转化为文本之外,部分研究考虑直接建模文本与图像的相关性,通过过滤无关模态信息提升情感分析准确性,Ju等人设计了跨模态关系检测辅助模块,判断图像是否能为目标方面的情感判断提供额外信息,进而动态调整模态融合权重[12]。同时,Yu等人则提出粗粒度-细粒度两级匹配策略,通过文本方面与图像区域的逐层匹配精准判断方面与图像的相关性[13],有效减少了无关视觉信息的干扰。此外,大型视觉语言模型(Large Vision-Language Models, LVLMs)的兴起为跨模态融合提供了新范式,最近,Feng等人利用Q-Former等视觉语言预训练模型进行指令微调,增强模型对方面相关跨模态信息的捕捉能力[14],进一步地,Wang等人通过融入世界知识丰富跨模态上下文,提升模型对隐含情感线索的推理能力[15],显著优化了复杂场景下的情感分析效果。针对跨模态信息对齐不精准的问题,为解决这一问题,Luwei等人首次将图像美学属性纳入多模态情感分析,通过提取视觉美学特征挖掘隐含情感线索[16],Chen等人应用最优传输理论设计了面向方面的视觉表征过滤机制,有效筛选与方面相关的视觉特征[17],受这些研究启发,Zhou等人则提出面向方面的跨模态信息检测方法,专门捕捉与目标方面相关的语义和情感信息[18],进一步提升了模态融合的针对性。

\subsection{图文关系推理}
图文关系推理是指在对文本、图像的内容进行建模并推断是否存在语义关联、情感关联的任务,其核心目标是明确不同模态信息的相互作用方式,为下游任务的提供支持。早期跨模态关系推理主要针对漫画、网页、科技文章等特定领域,Marsh等人提出了图像与文本关系的分类体系,为特定场景下的关系定义提供了基础[19]。然而随着社交媒体等开源多模态数据的增多,研究者开始聚焦通用场景的跨模态关系定义,为此,Chen等人基于图像与文本的视觉相似性定义相关性,通过图像与文本关键词的视觉特征匹配判断关联程度[20],同时,Vempala等人从全局语义角度出发,根据图像是否为文本提供额外信息来定义跨模态相关性[21],受这些研究启发,Wang等人则基于文本与图像的语义一致性构建关联模型[22]。近年来,隐式与情感语义层面的跨模态关系成为研究重点,Xu等人提出关联对齐网络,专门建模文本与图像之间的隐含相关性[23],Chen等人则聚焦情感语义对齐,通过计算整幅图像与文本的情感倾向一致性来定义跨模态关系[24]。
\section{研究内容}
本文从基于图文关系的视角,对多模态方面级情感分析展开研究。针对现有图文关系定义忽略图文间隐式情感关联的问题,本文基于情感表达的方面级图文相关性,构建了一个新的图文数据集以填补空白。在此基础上,本研究依次探索了基于图文情感线索引导的多模态方面级情感分析方法,以及基于多角度相关性解耦的模型增强策略。具体研究内容如下:

\textbf{(1)图文情感相关性语料构建与数据分析。}本研究从图文关系的视角出发,针对现有图文关系定义仅关注文本与图像的显式对齐或全局相关性,而忽略隐式情感关联,导致模型在复杂场景下易受干扰的问题,定义了基于情感表达的方面级图文相关性。通过人工标注3562条样本,构建了一个新的方面级图文相关性数据集。在此基础上,本研究对数据集进行了详细的统计分析,深入探究图文关系与情感分析之间的内在关联,为后续研究奠定了数据基础。同时,通过对比多种将跨模态相关性信息融入情感分类的方法,在多个数据集中验证了所提出的方面级跨模态相关性的有效性。

\textbf{(2)基于图文情感线索引导的多模态方面级情感分析}。尽管前述验证表明加入相关性信息能带来性能提升,但仅依靠粗粒度的相关性难以精确控制图文情感线索的贡献度,导致模型无法准确捕捉模态间的细粒度交互。因此,本研究提出了基于情感线索的分析方法。针对现有方法对情感线索建模粒度较粗、精度不足的问题,该方法利用多模态大模型提取与方面相关的情感线索描述,将其作为补充信息编码至上下文;同时利用相关性作为引导信号,实现情感线索与图文特征的细粒度自适应融合,有效抑制无关或冲突线索的干扰,从而提升模型对情感表达一致性的理解能力。

\textbf{(3)基于多角度图文相关性解耦的多模态方面级情感分析}。为了进一步提升模型对图文复杂关联的解析能力,本研究将方面级跨模态相关性解耦为语义与情感两个独立维度进行建模,以解决特征纠缠带来的理解偏差。具体而言,本研究提出了多角度图文相关性解耦的分析方法。该方法利用解耦后的多角度相关性特征作为门控引导信号,实现了图文特征在语义和情感层面的精确融合。此外,该方法将单模态情感线索从输入提示转化为训练阶段的监督信号,通过多任务学习机制促使模型主动捕获关键情感特征。

\section{组织结构}
本文围绕多模态方面级情感分析任务展开研究,总共包含五章,各章的内容具体组织结构如下:

\textbf{第一章 绪论}

本章首先介绍了多模态方面级情感分析研究的背景与意义。随后,对相关领域的国内外发展现状进行了概述,指出了现有工作的不足和借鉴之处,最后,阐述了本研究的研究逻辑与研究内容,并展示了论文的主要组织结构
 
\textbf{第二章 图文情感相关性语料构建与数据分析}

本章首先介绍了基于情感表达的方面级图文相关性的定义,并人工标注了3562条数据,构建了一个新的方面级图文相关性数据集。在此基础上,为深入了解图文关系和情感分析之间的关联,本研究对构建的数据集进行了详细的数据分析,为后续研究提供了数据基础。同时本研究探索并对比了多种将跨模态相关性信息融入多模态方面级情感分类的方法,在多个数据集中证明了我们提出的方面级跨模态相关性的有效性。

\textbf{第三章 图文情感线索引导的多模态方面级情感分析}

本章首先介绍了基于情感线索的多模态方面级情感分析方法。考虑到现有方法对模态中包含的情感线索建模粒度较粗质量差,该方法通过多模态大模型从图片和文本中提取方面相关的情感线索描述,将其作为补充信息编码到上下文中,同时利用该相关性作为引导信号来细粒度使用情感线索和图文情感特征的自适应融合,从而抑制无关或冲突线索的干扰。从而提升模型对情感表达一致性的理解能力。

\textbf{第四章 多角度图文相关性解耦的多模态方面级情感分析}

本章首先介绍了多角度图文相关性解耦的多模态方面级情感分析方法。该方法通过将跨模态相关性信息解耦为多个方面级相关性,将图文关系解耦为语义与情感两个独立维度进行建模;利用解耦后的多角度相关性特征作为门控引导信号,实现图文特征在语义和情感层面的精确融合。同时,该方法还考虑到将单模态情感线索从输入提示转化为训练阶段的监督信号,通过多任务学习让模型主动理解并捕获情感特征。


\textbf{第五章 总结与展望}

本章首先对本研究的研究内容进行了总结,指出了本研究的主要贡献和创新点。随后,对下一步的研究工作方向进行了展望。