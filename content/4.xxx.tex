\chapter{第四章xx}
第三章提出的图文情感线索引导的多模态情感分析通过引入细粒度的方面级跨模态情感相关性标签和单模态情感线索描述,进一步提升了多模态方面情感分类的性能,然而对于方面级跨模态情感相关性的利用中,我们认为现有的相关性同时考虑了情感相关性和语义相关性,但是针对于图像特征的使用并没有细粒度进行控制区分,因此我们提出将方面级跨模态情感相关性进行解耦为两个角度情感相关性和语义相关性,来分别控制图像中的语义特征和图像中的情感特征的细粒度使用,进一步增强多模态方面级情感分类的性能。同时,我们提出了单模态情感监督增强的预热训练方式,来帮助模型能够更好地理解文本和图像本身的情感。
\section{引言}

第三章提出RC-EmoCLue方法通过引入细粒度的方面级跨模态情感相关性标签和单模态情感线索描述,有效的帮助模型理解图文之间的情感语义关联,进一步提升了多模态方面情感分类的性能。
然而其对于方面级跨模态情感相关性的利用是将其分为三种类别:情感相关、语义相关、图文无关,我们认为提出方面级跨模态情感相关性同时考虑了情感角度和语义角度的图文相关关系,但是借助相关性对于图像特征的使用并从这两个角度进行细粒度地控制区分。
本章考虑这个角度去进一步增强第三章的方法,通过挖掘更细分的方面级跨模态情感相关性,实现对于图像情感和语义特征的细粒度使用,从而提升模型对于图文内容的情感信息和语义信息的充分理解。基于该思路,我们将方面级跨模态情感相关性解耦为两个角度的相关性:情感相关性和语义相关性,来分别控制图像语义特征和图像情感特征的细粒度使用。对于图像情感特征和图像语义特征的提取,我们使用Q-former设置针对图像情感和图像语义的prompt来得到对应图像细粒度特征。同时,我们提出了两阶段的训练方式,首先是单模态情感监督增强的预热,通过将文本情感和图像情感作为多任务来帮助模型更好地理解文本和图像本身的情感。接着是多模态方面级情感分类和多角度相关性的多任务学习,同时训练两个任务,将多角度相关性输出的情感相关性和语义相关性作用于对应的图像特征,实现细粒度使用。

本章提出的多角度多任务框架在twitter-15和twitter-17数据集的实验结果表明是有效的
\section{xxxx的多模态情感分析}
% 多任务定义、多角度相关性定义
% 输入的样本 $\mathbf{x}$ 包含一条推文,该推文由文本-图像对及其对应的方面组成。其中,$x^t$、$x^i$ 和 $x^a$ 分别表示文本、图像和方面。
% 特征抽取:
% 1. 文本特征抽取:使用t5模型,将prompt,句子,方面对拼接作为输入,输入到t5模型中,获得文本的embedding表示。对应符号表示为$h^t = t5(x^t, x^a)$。
% 2. 图像特征抽取:使用Q-former模型,我们使用两种不同的prompt,分别是情感prompt和语义prompt,来抽取图像中方面对应的语义特征和图像整体的情感特征。对应的符号表示为$h^s = Q-former(x^i, p^s)$和$h^e = Q-former(x^i, p^e)$,其中$p^s$和$p^e$分别是情感prompt和语义prompt。
% 多任务:
% 1. 文本情感监督:输入:$\mathbf{x} = (x^t, x^a)$,输出:情感标签 $y^t \in \{\textit{positive},  \textit{neutral},  \textit{negative}\}$,相关的表示和逻辑流程:prompt拼接,输入到t5模型,输出情感标签。目的是让模型理解文本中方面的情感信息。
% 2. 图像情感监督:输入:$\mathbf{x} = (x^i, x^a)$,输出:情感标签 $y^i \in \{\textit{positive},  \textit{neutral},  \textit{negative}\}$,相关的表示和逻辑流程:prompt拼接输入t5中获得embedding表示,然后对于提取抽取得到的情感特征,使用一个全连接层映射再和t5的输入进行拼接,最后输入到t5模型中,输出情感标签。目的是让模型理解图像中方面的情感信息。
% 3. 多角度相关性监督和融合,输入:$\mathbf{x} = (x^t, x^i, x^a)$,输出:情感相关性 $r^s \in \{\textit{relevant}, \textit{irrelevant}\}$ 和语义相关性 $r^e \in \{\textit{relevant}, \textit{irrelevant}\}$,相关的表示和逻辑流程:prompt拼接,输入到t5模型中,对于语义相关性,我们将抽取到的语义特征和文本特征进行拼接,然后输入到t5模型中,输出语义相关性。对于情感相关性,我们将抽取到的情感特征和文本特征进行拼接,然后输入到t5模型中,输出情感相关性。目的是让模型理解图文对中方面相关的情感相关性和语义相关性。
% 4. 多模态方面级情感分类,输入:$\mathbf{x} = (x^t, x^i, x^a)$,输出:情感标签 $y \in \{\textit{positive},  \textit{neutral},  \textit{negative}\}$,相关的表示和逻辑流程:prompt拼接输入到t5模型中,然后对于计算得到情感相关性和语义相关性作为门控权重控制图像情感特征和语义特征的细粒度使用,接着我们将图像情感特征、图像语义特征还有T5文本输入拼接再一起,然后输入到t5模型中,输出情感标签。

% 训练过程:
% 1. 单模态情感监督增强的预热训练:我们首先使用文本情感监督和图像情感监督作为多任务,来帮助模型更好地理解文本和图像本身的情感。训练过程中,我们使用交叉熵损失函数来优化模型参数。
% 2. 多模态方面级情感分类和多角度相关性的多任务学习:在预热训练完成后,我们将多模态方面级情感分类和多角度相关性监督和融合作为多任务,来训练模型。训练过程中,我们使用交叉熵损失函数来优化模型参数。

\section{实验设置与结果分析}
\subsection{对比实验设置与结果分析}
% 超参数设置:学习率、训练过程、批量大小、优化器、损失函数等。
\subsection{主要实验结果}
\subsection{消融实验分析}
\subsection{样例分析}
\section{本章小结}  